\documentclass{article}
\usepackage[utf8]{inputenc}
\usepackage{amsmath}
\usepackage{amsfonts}
\usepackage{graphicx}
\usepackage{hyperref}
\usepackage{listings}
\usepackage{longtable}

\title{Implementação do Algoritmo MD5 em Python}
\author{Ananda Guedes do Ó}
\date{Dezembro de 2024}

\begin{document}

\maketitle

\section*{Introdução}

O algoritmo MD5 (Message-Digest Algorithm 5) é amplamente utilizado para verificar a integridade de dados, sendo um dos métodos de hash mais conhecidos. Apesar das vulnerabilidades conhecidas, como a suscetibilidade a colisões, o MD5 ainda possui relevância educacional para estudar conceitos fundamentais de hashing e criptografia. Este trabalho apresenta a implementação do algoritmo MD5 em Python, detalhando suas etapas principais e realizando uma análise comparativa entre os resultados gerados manualmente e os gerados pela biblioteca hashlib. Além disso, este estudo investiga a sensibilidade do MD5 a pequenas modificações nas entradas e avalia o desempenho da implementação.

\section*{Objetivos}

O principal objetivo deste trabalho é desenvolver uma implementação do algoritmo MD5 em Python, abordando as seguintes metas específicas:

\begin{itemize}
    \item Implementar o algoritmo MD5 manualmente, respeitando as etapas teóricas descritas no funcionamento do MD5:
    \begin{itemize}
        \item Preenchimento (Padding): Adicionar bits de preenchimento e o tamanho original da mensagem.
        \item Divisão em Blocos: Dividir a mensagem em blocos de 512 bits.
        \item Processamento dos Blocos: Realizar as operações de mistura para cada bloco.
        \item Cálculo do Hash Final: Concatenar os valores dos registradores para gerar o hash de 128 bits.
    \end{itemize}
    \item Validar a implementação, comparando os hashes gerados com a função hashlib do Python.
    \item Demonstrar a imutabilidade do hash, observando mudanças drásticas com alterações mínimas nas entradas (propriedade de avalanche).
    \item Avaliar o desempenho da implementação em termos de tempo de execução.
\end{itemize}

\section*{Metodologia}

\subsection*{Estrutura do Código}

A implementação do algoritmo MD5 foi organizada em uma classe Python chamada MD5. Essa classe encapsula o processo completo de hashing, que inclui todas as etapas desde a inicialização até o cálculo final do hash. A estrutura da classe é descrita a seguir.

\subsubsection*{1. Inicialização}

A inicialização do MD5 define as constantes essenciais para o funcionamento do algoritmo:

\begin{itemize}
    \item Buffers Iniciais: O MD5 utiliza quatro registradores AA, BB, CC e DD, com valores iniciais específicos.
    \item Tabela de Constantes (K): Cada uma das 64 rodadas do MD5 usa um valor da tabela K, derivada da parte fracionária da raiz cúbica dos primeiros 64 números primos.
\end{itemize}

\subsubsection*{2. Função de Rotação Cíclica}

O MD5 faz uso da rotação cíclica de bits em suas operações de mistura. A função \texttt{\_left\_rotate} implementa essa rotação, deslocando os bits de um número para a esquerda de forma circular:

\begin{lstlisting}[language=Python]
def _left_rotate(self, x, c):
    # Realiza uma rotação cíclica dos bits de x em c posições.
    return ((x << c) | (x >> (32 - c))) & 0xFFFFFFFF
\end{lstlisting}

\subsubsection*{3. Preenchimento e Divisão em Blocos}

O processo de preenchimento (padding) adiciona um bit 1 seguido de bits 0 até que o comprimento total da mensagem seja congruente a 448, módulo 512. Após o preenchimento, adiciona-se o comprimento original da mensagem em bits como um valor de 64 bits. A mensagem é então dividida em blocos de 512 bits, que são processados individualmente.

\subsubsection*{4. Processamento dos Blocos}

Cada bloco de 512 bits é processado em 64 rodadas, usando funções de mistura e constantes derivadas. As operações realizadas são não lineares e envolvem a rotação cíclica, realizando modificações nos registradores AA, BB, CC e DD a cada rodada.

\subsubsection*{5. Cálculo do Hash Final}

Ao final do processamento de todos os blocos, os valores finais de AA, BB, CC e DD são somados aos valores anteriores dos buffers. O resultado final é o hash de 128 bits.

\section*{Resultados}

\subsection*{Validação}

Os resultados da implementação manual foram comparados com os valores gerados pela biblioteca hashlib do Python. A tabela a seguir mostra os valores de hash gerados para diferentes entradas:

\begin{longtable}{|l|l|l|l|}
\hline
\textbf{Método} & \textbf{Mensagem} & \textbf{Hash} & \textbf{Coincidem?} \\
\hline
\endfirsthead
\hline
Manual & b"Hello World" & b10a8db164e0754105b7a99be72e3fe5 & Sim \\
\hline
Hashlib & "Hello World" & b10a8db164e0754105b7a99be72e3fe5 & Sim \\
\hline
\end{longtable}

\subsection*{Sensibilidade}

Alterações mínimas na entrada resultaram em grandes mudanças no hash, confirmando a propriedade de avalanche do MD5. A tabela a seguir ilustra esse comportamento:

\begin{longtable}{|l|l|}
\hline
\textbf{Mensagem} & \textbf{Hash} \\
\hline
\endfirsthead
\hline
b"Hello" & 5d41402abc4b2a76b9719d911017c592 \\
\hline
b"Hello00" & bc6e6f16b0e76e76b70bff04ff872d5c \\
\hline
\end{longtable}

\section*{Discussão}

Durante a implementação manual do algoritmo MD5, surgiram diversos desafios técnicos que exigem uma boa compreensão dos detalhes do funcionamento do algoritmo e da matemática envolvida. Entre os principais desafios, destacam-se:

\begin{itemize}
    \item Tratamento de Erros e Exceções
    \item Compreensão dos Detalhes Matemáticos
    \item Desempenho e Eficiência
\end{itemize}

\section*{Conclusão}

A implementação manual do algoritmo MD5 foi um exercício educativo valioso, permitindo uma compreensão aprofundada dos detalhes técnicos e matemáticos envolvidos no funcionamento do algoritmo de hash. A implementação foi validada com sucesso ao gerar hashes que coincidem com os valores produzidos pela biblioteca hashlib, demonstrando a exatidão do processo. A propriedade de "avalanche", onde pequenas mudanças na entrada resultam em grandes diferenças no hash, foi claramente observada, o que reforça a sensibilidade do algoritmo a alterações mínimas.

No entanto, ao comparar a implementação manual com a versão otimizada da hashlib, ficou claro que a solução manual não é eficiente em termos de desempenho. A execução de operações bit a bit repetitivas, somada à necessidade de tratamentos de erro e ao manuseio detalhado de dados, torna a versão manual mais lenta, especialmente quando processa grandes volumes de dados.

A principal limitação do MD5 é sua vulnerabilidade a colisões, o que o torna inadequado para aplicações que exigem alta segurança.

\section*{Referências}

\begin{itemize}
    \item IETF. RFC 1321 - MD5 Message-Digest Algorithm. Abril de 1992. Disponível em: \url{https://www.ietf.org/rfc/rfc1321.txt}. Acesso em: 30 nov. 2024.
    \item PANTSMAN0. Código base dividido em arquivos C. Disponível em: \url{https://github.com/pantsman0/md5/blob/master/md5c.c}. Acesso em: 1 dez. 2024.
    \item KINSTA. Utilizando hash em Python. Disponível em: \url{https://kinsta.com/pt/blog/hashing-python/}. Acesso em: 28 nov. 2024.
    \item OVERLEAF. Plataforma para gerar o relatório SBC. Disponível em: \url{https://pt.overleaf.com/latex/templates/sbc-conferences-template/blbxwjwzdngr}. Acesso em: 5 dez. 2024.
\end{itemize}

\end{document}
